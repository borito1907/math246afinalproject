\documentclass{article}
\usepackage{geometry}
\usepackage{graphicx} % Required for inserting images
\usepackage{amsmath, amsthm, amssymb}
\usepackage{parskip}
\usepackage{xr}
\usepackage{textcomp}
\usepackage{hyperref}

\newgeometry{vmargin={15mm}, hmargin={24mm,34mm}}
\theoremstyle{definition} 
\newtheorem{definition}{Definition}

\newtheorem{theorem}{Theorem}[section]
\newtheorem{lemma}[theorem]{Lemma}
\newtheorem{corollary}{Corollary}[theorem]
\newtheorem{proposition}[theorem]{Proposition}
\newtheorem{example}[theorem]{Example}

\newcommand{\mytilde}{\raisebox{0.5ex}{\texttildelow}}
\newcommand{\N}{\mathbb{N}}
\newcommand{\Z}{\mathbb{Z}}
\newcommand{\R}{\mathbb{R}}
\newcommand{\Q}{\mathbb{Q}}
\newcommand{\C}{\mathbb{C}}
\newcommand{\logderivative}[1]{\frac{#1'}{#1}}
\newcommand{\riemannsphere}{\C \cup \{ \infty \}}

\title{Complex Analysis}
\author{Boran Erol}

\begin{document}

\maketitle

\section{Introduction}

The Prime Number Theorem is one of the most celebrated theorems in mathematics. In this writeup,
we look at some of the history of the Prime Number Theorem and review Newman's proof with a little bit 
more detail and references to our class this quarter. At the end,
I will give a brief overview of Alan Turing's relationship to the Riemann Hypothesis
and point out an interesting connection to a remark made in class this quarter.

Let $ \pi(x) $ be the function that counts the primes up to $ x $, i.e.\[ \pi(x) := \text{number of primes less than or equal to $ x $} .\]

The Prime Number Theorem states that $ \pi(x) \sim \frac{x}{\log x} $, where $\log $ denotes the natural logarithm.
Here, $ \sim $ means that \[ \lim_{x \to \infty} \frac{ \pi(x)}{x/ \log{x}} = 1 \]

In other words, the Prime Number Theorem states that the $n$th prime will have size roughly $ n \log n $.
In fact, Barkley Rosser proved in 1939 that the nth prime number is strictly greater than $ n \log n $ \cite{rosser_n-th_1939}.

The Prime Number Theorem was first conjectured by Gauss when he was around 15 (in fact, 
he conjectured that $ \pi(x) \sim Li(x) $, where $ Li(x) $ is the \href{https://en.wikipedia.org/wiki/Logarithmic_integral_function}{logarithmic integral function}, which turns out to be a better approximation) 
using prime number tables compiled by Johann Heinrich Lambert, who is
known for proving that $ \pi $ is irrational \cite{klyve_origin_2018} \cite{noauthor_histoire_nodate}. Legendre, seemingly independently from Gauss,
guessed in 1808 that

\[ \pi(x) \sim \frac{x}{\log x - A(x)} \]

for some $ A(x) $ that tends to a constant as $ x $ goes to infinity \cite{bateman_hundred_1996}. In fact,
earlier in his writing, Legendre even specifies this constant, stating 

\[ \pi(x) \sim \frac{x}{\log x - 1.08366} \]

instead \cite{bambah_centennial_2000}, but no explanation for this constant has been found in Legendre's notes.
Following Gauss and Legendre, Chebyshev came along and proved that $ \pi(x) = \Theta(x) $,
and provided extremely good constants. More specifically, Chebyshev \cite{bateman_hundred_1996} proved that for 
sufficiently large $ x $ we have that

\[ 0.92 \frac{x}{\log x} \leq \pi(x) \leq 1.10 \frac{x}{\log x}. \]

In fact, Newman's short proof \cite{zagier_newmans_1997} still uses Chebyshev's upper bound
(or at least the idea behind the upper bound). We'll skip the proof of the upper bound, because
it doesn't relate to complex analysis and also because it isn't the interesting portion of the proof.
You can see the upper bound in virtually every reference mentioned in this writeup, or \href{https://artofproblemsolving.com/wiki/index.php/Chebyshev_theta_function#Estimates_of_the_function}
{this AoPS page}.

Moreover, Chebyshev's arguments use elementary tricks using the binomial coefficients,
so the `hardness' of the Prime Number Theorem is proving that the $ \pi(x) \sim \frac{x}{\log x} $,
not $ \pi(x) = \Theta(\frac{x}{\log x}) $.

In 1859, Riemann came along and published the paper that introduces the
Riemann Hypothesis in an 8-page paper \cite{wilkins_number_1998}. In the paper, Riemann also
mentions the conjecture that $ \pi(x) < Li(x) $ (though it's not certain whether he believed it), which would later
be disproved by Littlewood and capture Alan Turing's interest \cite{princetonacademics_andrew_2012} \cite{matiyasevich_riemann_2020} \cite{hejhal_alan_2012}. More importantly,
Riemann considers the Zeta function first introduced by Euler in 1737 
with complex inputs and proves that proving the Prime Number Theorem
reduces to showing that the Zeta function doesn't have any zeros with $ Re(s) = 1 $.

In 1896, approximately a 100 year after Gauss's conjecture, Hadamard and de la Vallée Poussin
independently proved the Prime Number Theorem using Riemann's proposed method. De la Vallée Poussin's
initial solution was apparently incredibly messy, and he also admits that his solution was worse than
Hadamard's \cite{bambah_centennial_2000}. Over the next 100 years, the proofs would be improved with new tricks and observations, 
and in 1980 Newman discovered the short proof that we present in this writeup.

In 1949, Selberg and Erdös (there's a priority dispute, and Richard Borcherds thinks
that Selberg wins this dispute since Erdös used Selberg's identity to prove the result, which is the
key ingredient, but this is obviously a subjective matter) proved the Prime Number Theorem using non-analytic methods \cite{bambah_centennial_2000} \cite{richard_e_borcherds_introduction_2022}.

\subsection{Overview}

Let's now introduce some notation. Throughout, $ s $ will denote a complex number with $ s = \sigma + it $ and $ \sigma, t \in \mathbb{R} $.
Moreover, $ p $ will denote a prime and $ \rho $ will denote a zero of $ \zeta(s) $.
When we use the notation 

\[ \sum_{p}, \sum_{\rho}\]

this means summing (or multiplying) over all primes or the zeros of $ \zeta $, respectively.
There are many resources online that try to explain the Prime Number Theorem and many posts
on MSE that try to clarify the proof \cite{bandeira_complex_nodate} \cite{avi_why_2015} \cite{coffee_table_what_2018}. 

As I was exploring these resources myself to try to
understand the Prime Number Theorem, I realized that (like me) most beginners
had trouble understanding the so-called `Analytic Theorem' in Newman's paper.

Thus, after defining the zeta function and proving Euler's Identity, I'll
make some handwavy arguments relating the Analytic Theorem in Newman's paper
to the Ikehara-Wiener Theorem, and see how the Analytic Theorem in Newman's paper
is a simplification of this theorem for which we can develop some intuition.

Then, I will tie this back to the Riemann Zeta function, thus reducing the proof
to proving certain properties of the Zeta function. 

\section{The Basic Objects and Their Properties}

Let's now define the star of the show, the Riemann Zeta function and
try to understand how it relates to prime numbers.

Let \[ \zeta(s) := \sum_{n = 1}^{\infty} \frac{1}{n^{s}} .\] 

Notice that for $ s $ with $ Re(s) > 1 $ we have that the infinite series is absolutely
summable and therefore convergent. This is because if the infinite series is absolutely
summable, then the real and imaginary parts are absolutely summable, which implies that
the real and imaginary parts of the series converge, which implies that the complex-valued
sequence of partial sums converges. Also notice that $ \zeta(s) $ has a simple pole
at $ s = 1 $ since the harmonic series diverges. 

Let us now immediately relate this function to primes by proving Euler's Identity:

\begin{lemma}
    For all $ s \in \C $ such that $ Re(s) > 1 $, we have that $ \zeta(s) = \prod_{p} \frac{1}{1 - p^{-s}} $.
\end{lemma}

\begin{proof}
    
\end{proof}

Notice that the product formula also implies that $ \zeta(s) \neq 0 $ if $ Re(s) > 1 $ since
none of the terms in the infinite product is zero.

Let us consider the logarithmic derivative of $ \zeta(s) $.
Recall that in class, we mentioned that the log derivative of a finite product is the sum of the
log derivative of the individual terms. This can in fact be extended to infinite products
by considering infinite product as infinite sums of logarithms. Taking this for granted,
let's now see what the derivative of a single term looks like.


\[ \frac{d}{ds} (1 - p^{-s})^{-1} = - (1 - p^{-s})^{-2} \log(p) p^{-s}\]

Thus, 

\[ \frac{\frac{d}{ds} (1 - p^{-s})^{-1}}{(1 - p^{-s})^{-1}} = \frac{\log(p) p^{-s}}{1 - p^{-s}} = - \frac{\log (p)}{p^{s} - 1} \]

Then, 

\[ \frac{\frac{d}{ds} \zeta(s)}{\zeta(s)} = - \sum_{p} \frac{\log (p)}{p^{s} - 1} \]


Now, I'll first wave my hands for two paragraphs and give an intuitive argument
using the residue theorem, and then I'll compare this argument to a rigorous
but super counterintuitive proof.

Notice that the right-hand side of this equation is counting the number of primes, albeit
weighted a in a little bit of a weird way. However, we can also see the right-hand side the as
`essentially' the following infinite series (notice that I've already starting waving my hands
by omitting the $ -1 $ from the denominator):

\[ \sum_{n} \frac{\Lambda(n)}{n^{s}}, \]

where $ \Lambda(n) = \log p $ if $ n $ is the power of a prime number and $ 0 $ otherwise.

This is called a \textbf{Dirichlet series} and $ \Lambda(n) $ is called the \textbf{von Mangoldt function}.
In fact, we'll call the function representing this series $ \Phi(s) $.

It turns out that in 1896, Mellin figured out how to get back the growth of a Dirichlet series
in terms of its transform, the so called Mellin transform. Then, Wiener and Ikenara
took this idea and proved the following theorem in 1932 \cite{korevaar_newmans_1982}:

\begin{theorem}
    Let $ f(x) $ be non-negative and nondecreasing on $ [1,\infty) $ such that the Merlin
    transform

    \[ g_{0}(s) = - f(1) + s \int_{1}^{\infty} f(x) x^{-s - 1} dx \]
    
    exists for $ Re(s) > 1 $. Suppose that for some constant $ c $, the function 
    \[ g_{0}(s) - \frac{c}{s-1} \]

    has a continuous extension to the closed half-plane $ Re(s) \geq 1 $. Then, $ \frac{f(x)}{x} \to 1 $
    as $ x \to \infty $.
\end{theorem}

I'm just putting this statement here for intuition, so I'm not going to explain this statement in detail.
All I want to say is that it intuitively makes sense that $ \Phi(s) $ will play the role of $ g(s) $
in this statement and $ \Lambda(x) $ will play the role of $ f(x) $. And, you guessed it, $ c $ will
turn out to be $ 1 $.

We'll prove an easier version of this statement (\cite{korevaar_newmans_1982} calls it `A Poor Man's Ikehara-Wiener Theorem')
which is sufficient to prove the Prime Number Theorem.

But now, notice that in order to use this theorem, we need to show that $ \Phi(s) - \frac{1}{s - 1} $
can be continued analytically on the real line. This is where the crucial observation of Riemann
comes in. Showing that $ \Phi(s) - \frac{1}{s - 1} $ is analytic on $ Re(s) \geq 1 $, using the Residue
Theorem, is equivalent to proving that $ \zeta(s) \neq 0 $ for any $ s $ with $ Re(s) = 1 $.
Riemann stated this in his 1859 paper, but couldn't prove this result.

To sum it up in order to understand the coefficients of a Dirichlet series, we need to integrate
the Dirichlet series. But since this particular Dirichlet series turns out to be the 
logarithmic derivative of the Zeta function, it suffices to understand integrals of the
logarithmic derivative Riemann Zeta function, where we can use the Residue theorem.

Here's the statement we'll prove instead of the Ikehara-Wiener Theorem:

\begin{lemma}
    Let $ f:[0,\infty) \xrightarrow{} \C$ be a bounded and locally integrable function.
    Define 

    \[ g(z) := \int_{0}^{\infty} f(t) e^{-tz} dt, Re(z) > 0 \]

    Assume $ g(z) $ extends holomorphically to $ Re(z) \geq 0 $.
    Then, $ \int_{0}^{\infty} f(t) dt $ exists and equals to $ g(0) $.
\end{lemma}

To not break up the flow of the argument, I'll defer the proof of this statement to the appendix.
However, notice the similarities between this theorem and the Ikehara-Wiener Theorem.
In both theorems, we understand the asymptotic behavior of $ f(x) $ by using the
niceness properties of a Fourier-like integral. In fact, in our proof, we won't even need
the value $ g(0) $: convergence will be sufficient to deduce the prime number theorem.

Let's now start the proof.

\section{Overview of the Proof}

Let $ \nu(x) $, which is sometimes called the \textbf{Chebyshev's theta function},
be defined as 

\[ \nu(x) = \sum_{p \leq x} \log p \]

Intuitively, this function
counts each prime and weights them according to the inverse of their density.
Thus, we expect $ \nu(x) \sim x $ if we believe the Prime Number Theorem.
In fact, this turns out to imply the Prime Number Theorem. I believe that this
is kind of clear, so I have relegated this implication to the appendix.

\begin{lemma}
    If $ \nu(x) \sim x $, $ \pi(x) \sim x \log x $.
\end{lemma}

Thus, if we prove this lemma, we reduce the prime number theorem to proving $ \nu(x) \sim x $.

In order to do this, we'll show that the integral 

\[ \int_{1}^{\infty} \frac{\nu(x) - x}{x^{2}} \]

converges. Let's see why this would intuitively be sufficient. By Chebyshev's upper bound,
we know that $ \nu(x) = O(x) $. Now, notice that if $ \nu(x) $ wasn't exactly approaching $ x $,
the above integral would turn into the harmonic series and would blow up to infinity. Thus, this is intuitively
sufficient to show that $ \nu(x) $ acts exactly like $ x $. In order to prove that this
integral is convergent, we'll use the Poor Man's Ikehara-Wiener Theorem!

More specifically, we 'll pick $ f(t) := \nu(e^{t}) e^{-t} -1 $ and 
\[ g(z) = \frac{\Phi(z+1)}{z+1} - \frac{1}{z} \]

Well, these look like they have absolutely nothing to do with $ \frac{\nu(x) - x}{x^{2}} $,
but some trickery with integrals shows that they are actually equivalent. This
is done in $ \ref{lem:newman_v} $.

Thus, to apply the theorem, we'll need $ f(t) $ to be locally bounded 
and $ g(z) $ to be analytic on $ Re(z) \geq 0 $.
The fact that $ f(t) $ is bounded follows from the Chebyshev's upper bound,
i.e. from $ \nu(x) = O(x) $. The analyticity of $ g(z) $ is equivalent to proving
that the Zeta function has no zeros with $ Re(s) = 1 $, as per the Residue Theorem
intuition. The formal proof is given in \ref{lem:newman_v_conditions}.



Thus, we have reduced the proof of the Prime Number Theorem to meromorphically
extending the Zeta function to the right-half plane and showing that it doesn't
have any zeros with $ Re(s) = 1 $. We now prove these to establish the prime number theorem.

Since our argument was a bit convoluted, I also strongly encourage the reader to compare
the flow of this argument to Newman's paper. (DO I? I'M NOT SURE IF I DO. COME BACK
TO THIS AT THE END.)

\section{Properties of the Zeta Function}


\section{Alan Turing and the Riemann Zeta Function}

I come from a computer science background, and therefore I was fascinated to learn
that Alan Turing also did some work on the Riemann Zeta function. In fact, one of
his contributions involved developing the so-called Turing method to aid in the
numerical verification that obviates the use of the Argument Principle sketched in
the class.

Recall that during class we argued that we can use the argument principle and the fact
that the zeros of the Zeta function are symmetric across the critical line to verify
the Riemann hypothesis numerically.

The argument went as follows: use numerical methods to estimate the location
of a zero, use the Argument Principle to integrate
the ?? around the purported zero, and check whether the numerical integral
gives a value close to $ 1 $. Even though numerical integration is imprecise,
since the winding number is an integer, we can be certain that the zero is on
the critical line, since otherwise we would have two zeros.

Instead, Alan Turing proposed using the formula (that I don't really understand)

\[ N(t) = 1 + \frac{1}{\pi} \theta(t) + S(t) \]

where $N(t)$ counts the number of zeros in the critical strip with imaginary part 
in $ [0,t] $ and  $ \theta(t) $ is a smooth function called the Riemann-Siegel theta function.
Since $ N(t) $ isn't smooth and $ 1 + \frac{1}{\pi} \theta(t) $ is, $ S(t) $ should 
also have jump discontinuities to catch up with $ N(t) $. Moreover, if $ S(t) $ jumps 
by two at any point (unless the Zeta function has non-simple zeros, which, as stated in lecture,
would be super surprising for reasons beyond me), we would gather that the zero was not on the
critical line. I found this super interesting since it both pertained to something we talked about
in lecture and Alan Turing!

Another gem I found while looking into this is the following story: 
Alan Turing apparently got a 40 pound grant from the Royal Society during his time
in Cambridge to run computer simulations for 

Here's a quote from his notes:


So not only did Alan Turing not believe in the Riemann Hypothesis, he also believed that he
could find a counterexample with a pretty small imaginary value! Alan Turing's belief is confirmed by his
1953 paper, where he states

\newpage

$ \pi(x) $ is a bit unwieldy, so people have developed under functions to examine
$ \pi(x) $ that act a bit nicer and exhibit the `same' behavior asymptotically. Two
functions we'll use are $ \Phi(x) $, also called the , and $ \nu(x) $, also called
Chebyshev's theta function. Here are the definitions:

\[ \Phi(s) = \sum_{p \leq x} \frac{\log p }{p^{s}}\]






% \[ \{ M^{-1}\phi (x) \} (x) = \frac{1}{2 \pi i } \int_{c - i \infty}^{c + i \infty} x^{-s} \phi(s) ds \]

\bibliographystyle{plainurl} 
\bibliography{246a}   

\appendix

\section{A Poor Man's Ikehara-Wiener Theorem}

\section{Relationship of $ \Phi(s) $ to $ \frac{\zeta(s)^{\prime}}{\zeta(s)}$}

\begin{lemma}\label{lem:newman_v}

\end{lemma}



\begin{lemma}\label{lem:newman_v_conditions}
    $ \frac{\Phi(s + 1)}{s + 1} - \frac{1}{s} $ is analytic on 
    $ Re(s) \geq 0 $ if and only if $ \zeta(s) $ has no zeros
    with $ Re(s) = 1 $.
\end{lemma}
\begin{proof}
    Let's first prove the statement without the denominator.

    Notice that

    \[ \frac{\zeta(s)^{\prime}}{\zeta(s)} - \Phi(s) = \sum_{p} \log(p) (\frac{1}{p^{s} - 1} - \frac{1}{p^{s}}) \]
    
    Rearranging, we get
    
    \[ \frac{\zeta(s)^{\prime}}{\zeta(s)} = \Phi(s) + \log(p)[\frac{1}{p^{s}(p^{s} - 1)}] \]
    
    More specifically, let $ x = e^{t} $
    and notice that 
    
    $ f(x) = \frac{\nu(x)}{x} - 1 $, so $ f(x) \leq cx - 1 $ for some $ c > 0 $.
\end{proof}



\end{document}
